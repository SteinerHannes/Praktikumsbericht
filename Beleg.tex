\newcommand{\cmmnt}[1]{}
% Bei Abgabe löschen
\documentclass[nomenclature, 150]{HSMW-Thesis}

\Art{Praktikumsbericht}

\Anrede{Herr}
\Vorname{Hannes}
\Nachname{Steiner}

\Thema{Entwicklung eines Dokumenten Scanners}
\Unterthema{Digitalisierung der Verwaltung an der Hochschule Mittweida}

\Studiengang{Softwareentwicklung}
\Seminargruppe{IF17wS-B}
\Fakultaet{Angewandte Computer- und Biowissenschaften}

\Erstpruefer{Prof\@. Dr\@. Mark Ritter}
\Zweitpruefer{}

\Datum{13.03.2020}

\Tag{13}
\Monat{März}
\Jahr{2020}

\Anlagen{}
\Copyright{Dieses Werk ist urheberrechtlich geschützt.}
\Textsatz{}
\Druck{}
\Verlag{}
\ISBN{}

\begin{document}

\begin{Referat}
% Referat
\end{Referat}

%\begin{Vorwort}
%% Vorwort
%\end{Vorwort}

\Hauptteil
% Diese Anweisung nicht loeschen!

\chapter{Einleitung}
	Digitalisierung wird oft als Integration von digitaler Technologie in den Alltag verstanden, und soll helfen Zeit einzusparen \cite{digital}. Mit diesem Gedanken initiierten die Mitarbeiter Holger Langner und Falk Schmidsberger der Hochschule Mittweida, das Projekt \textit{Memo Space}. Im Zuge dessen sollen kleinere Forschungsergebnisse entstehen, die richtungsweisend für die Digitalisierung der Verwaltung von Lehr- und Forschungseinrichtung sind.

	Eine der ersten Ideen \cmmnt{im Zusammenhang mit Memo Space} ist es, die Arbeit von Klausur-Prüfern zu erleichtern. Diese müssen, nachdem die Klausuren kontrolliert wurden, die Benotungen, sowie die Eckdaten der Studenten, in ein digitales Format bringen. Grund dafür ist, dass die Noten in das Notensystem der Einrichtung eingetragen werden müssen. 

	Im Rahmen eines Forschungspraktikums an der Hochschule Mittweida arbeiteten der Student Tobias Kallauke und der Verfasser, gemeinsam an einer Lösung zur Digitalisierung dieses Arbeitsschrittes. ...

\chapter{Problemstellung} 
	Hochschulmitarbeiter sitzen zum Ende eines Semesters über Tage an der Kontrolle von Klausuren. Diese Aufgabe muss stets mit hoher Konzentration erledigt werden, und lässt sich aber in den meisten Fällen nur schwer durch Maschinen ersetzen. Unter keinen Umständen dürfen bei der Bewertung Fehler vorkommen, was jedoch bei der kognitiven Last des Prüfers immer wieder passiert. Auch nach der Durchsicht der Prüfungsaufgaben ist eine hohe Achtsamkeit wichtig. Denn anschließend wird die Benotung in eine digitale Tabelle geschrieben. In diese muss die Matrikelnummer, der Vor- und Nachname, sowie die Note des Studenten eingetragen werden. Hier kommt es vor allem bei der Matrikelnummer und der Zensur auf die Richtigkeit jedes Zeichens drauf an.
	Das Ziel ist es 


\chapter{Anforderungen}


\chapter{Konzept}
%	Die Grund Idee des Dokumentenscanner ist es den Vorgang der Klausuren-Kontrolle und deren Noteneintragung zu vereinfachen. Genauer soll es möglich sein die wichtigen Daten auf der ersten Seite einer Klausur, wie Vor- und Nachname des Studeten, seine Matrikelnummer sowie die Note zu erkennen, digitalisieren und in ein geeignetes Format zu bringen, um es anschließend der Notenfreigabe der Hochschule zu übermitteln. 
%
%	Des Weiteren ist es für die Prüfer der Klausur hilfreich, wenn die Punkte der Klausur kontrolliert bzw. überprüft werden. Konkret bezieht sich die Überprüfung der Punkte auf eine Klausuren-Format-Vorlage, die im Bereich der Fakultät Angwandte Computer- und Biowissenschaften genutzt wird. Diese Vorlage hat auf dem Deckblatt der Klausur zu jeder Aufgabe ein Feld für die erreichten und die zu erreichenden Punkte. Sowie die Summe der erreichten Punkte. Hier könnte der Dokumentenscanner die erreichte Gesamtpunktezahl sowie die daraus resultierende Note überprüfen. Des Weiteren befinden sich auf den nachfolgenden Seite der Klausur über jeder Aufgabe ein Feld für die erreichten und die zu erreichenden Punkte der jeweiligen Aufgabe. Die dort vom Prüfer eingetragenen Punkte könnten ebenfalls mit denen auf dem Deckblatt verglichen werden. 
%
%	Durch die Erweiterung der Grundidee um Scan-Vorlagen wurde es auch möglich Daten anderer Dokumente zu digitalisieren und in ein geeignetes Format zu bringen. Genauer soll es dem Benutzer möglich sein von jedem beliebigen Dokument eine Vorlage zu erstellen, in der er Stellen auf dem Dokument markiert, wo sich die zu digitaliserenden Daten befinden.


\chapter{Grenzen}
	Die Abkürzung etc.\nomenclature{etc.}{et cetera} steht im Abkürzungsverzeichnis.

\Anhang

%\chapter{}

\begin{thebibliography}{99}
	\bibitem{digital}
		Michael Graf, Partner bei PwC
		\\\texttt{http://www-cs-faculty.stanford.edu/\~{}uno/abcde.html}
		
\end{thebibliography}

\end{document}

