\newcommand{\cmmnt}[1]{}
% Bei Abgabe löschen
\documentclass[nomenclature, onesided, 150]{HSMW-Thesis}

\usepackage{hyperref}
\hypersetup{
	urlcolor=black
}
\usepackage[utf8]{inputenc}
\usepackage[T1]{fontenc}
\usepackage[ngerman]{babel}
\usepackage{hyphenat}

\Art{Praktikumsbericht}

\Anrede{Herr}
\Vorname{Hannes}
\Nachname{Steiner}

\Thema{Entwicklung einer Dokumenten-Scanner-App}
\Unterthema{Digitalisierung der Verwaltung an der Hochschule Mittweida}

\Studiengang{Softwareentwicklung}
\Seminargruppe{IF17wS-B \\[\bigskipamount] Matrikelnummer:\\ 46540}
\Fakultaet{\MNI}

\Datum{13.03.2020}

\Tag{13}
\Monat{März}
\Jahr{2020}

\Anlagen{}
\Copyright{Dieses Werk ist urheberrechtlich geschützt.}
\Textsatz{}
\Druck{}
\Verlag{}
\ISBN{}


\begin{document}

\begin{Referat}
% Referat
In dem vorliegendem Praktikumsbericht...

\end{Referat}

%\begin{Vorwort}
%% Vorwort
%\end{Vorwort}

\Hauptteil
% Diese Anweisung nicht loeschen!

%%%%%%%%%%%%%%% - EINLEITUNG - %%%%%%%%%%%%%%%

\chapter{Einleitung}
	Digitalisierung wird gängig als Integration von digitaler Technologie in den Alltag verstanden, und soll helfen Zeit einzusparen \cite{digital}. Mit diesem Gedanken initiierten die Mitarbeiter Holger Langner und Falk Schmidsberger der Hochschule Mittweida, das Projekt \textit{Memo Space}. Im Zuge dessen sollen kleinere Forschungsergebnisse entstehen, die richtungsweisend für die Digitalisierung der Verwaltung von Lehr- und Forschungseinrichtung sind.

	Im Rahmen eines zwölfwöchigem Forschungspraktikums an der Hochschule Mittweida arbeiteten der Student Tobias Kallauke und der Verfasser des Berichts, gemeinsam an einem Forschnungsprojekt von Memo Space. Dabei entwickelten sie ein Software-System, mit dem die Arbeit von vielen Hochschulmitarbeitern erleichtert und auch Zeit eingespart werden soll.

%%%%%%%%%%%%%%% - HSMW - %%%%%%%%%%%%%%%

\chapter{Hochschule Mittweida}
	Die Hochschule Mittweida - university of applied science (HSMW) \nomenclature{HSMW}{Hochschule Mittweida - university of applied science} wurde vor über 150 Jahren gegründet. Heute lehrt und forscht sie mit ca. 6000 Studenten in fünf Fakultäten und vier Forschungsschwerpunkten\cite{noauthor_hochschule_nodate} . Eines der Schwerpunkte ist die angewandte Informatik, in dem Memo Space angesiedelt ist.
	
	- wie komme ich auf 5? ''Laut der Studie...''
	
	Pro Semester schreibt jeder Student ca. 5 Prüfungen, was bedeutet, dass im Jahr um die 60.000 Klausuren kontrolliert werden. Dazu kommt, dass nachdem die Klausuren kontrolliert wurden, die Zensur, sowie die Eckdaten der Studenten, die an der Prüfung teilgenommen haben, in ein digitales Format gebracht werden muss. Grund dafür ist, dass die Übertragung der Noten in das Notensystem der Hochschule. Da die Mitarbeiter der Fakultät \textit{Angewandte Computer- und Biowissenschaften} (Fakultät CB)\nomenclature{Fakultät CB}{Fakultät für angewandte Computer- und Biowissenschaften} Holger Langner und Falk Schmidsberger selbst Klausuren kontrollieren und die Problematik genau kennen, entstand hier eines der ersten Ideen für Memo Space.

%%%%%%%%%%%%%%% - PROBLEMSTELLUNG - %%%%%%%%%%%%%%%

\chapter{Problemstellung} 
	An der Kontrolle von Klausuren sitzen zum Ende eines Semesters Hochschulmitarbeiter über Tage dran. Diese Aufgabe muss stets mit hoher Konzentration erledigt werden, und lässt sich aber in den meisten Fällen nur schwer durch Maschinen ersetzen. Unter keinen Umständen dürfen bei der Bewertung Fehler vorkommen, was jedoch bei der kognitiven Last der Prüfer immer wieder passiert. Auch nach der Durchsicht der Prüfungsaufgaben ist eine hohe Achtsamkeit wichtig. Denn anschließend wird die Benotung in eine digitale Tabelle überführt. In diese muss die Matrikelnummer, der Vor- und Nachname, sowie die Note des Studenten eingetragen werden. Hier kommt es vor allem bei der Matrikelnummer und der Zensur auf die Richtigkeit jedes Zeichens drauf an. 
	
	\section{Digitalisieren der Klausur-Daten}
	Für genau diesen Vorgang des Digitalisierens wird eine Lösung gesucht. Die Prüfer sollen so bequem und möglichst zeitsparend diese Aufgabe verrichten, ohne dabei ihre Aufmerksamkeitsspanne zu überlasten. Des Weiteren müssen die Ergebnisse der Prüfungen, sowie die Eckdaten der Studenten in ein geeignetes digitales Format gebracht werden, um es der Notenfreigabe weiterzuleiten. Darüber hinaus empfiehlt es sich, digitale Kopien der Klausuren abzuspeichern, um sie nicht nur analog zu archivieren.
	
	\section{Klausuren-Vorlage}
	
	- Bild einfügen
	
	Eine Klausuren-Vorlage bzw.\nomenclature{bzw.}{beziehungsweise} ein Gestaltungsleitfaden für Klausuren der Fakultät \textit{Angewandte Computer- und Biowissenschaften} (Fakultät CB)\nomenclature{Fakultät CB}{Fakultät für angewandte Computer- und Biowissenschaften} bietet außerdem die Möglichkeit der Kontrolle des Prüfers an. Genauer ist es durch das vorgegebene Layout der Klausur möglich, einen Teil der Arbeit des Prüfenden auf Fehler zu untersuchen. Auf dem Deckblatt der Klausur ist eine Tabelle, mit drei Zeilen und für jede Aufgabe eine Spalte. In der ersten Zeile befinden sich die Nummern der Aufgaben. In der zweiten, die zu erreichenden Punkte der Aufgabe. Und in der dritten Zeile trägt der Prüfer die erbrachten Punkte des Studenten ein. Unter der Tabelle befindet sich ein Feld für die erreichte Gesamtpunktzahl, sowie ein Feld für die aus den Punkten resultierende Note. Die Punkte pro Aufgabe, die Gesamtpunktzahl und die Zensur stehen in (Korrelation?) Relation zu einander, sodass aus den Punkten der Aufgaben die beiden anderen Felder errechnet werden und mit den Ergebnissen des Prüfers abgeglichen werden könnten. Eine weiteres Merkmal der Klausuren-Vorlage ist ein Feld, für die vom Studenten erreichten Punkte über jeder Aufgabenstellung. Die dortige Angabe sollte mit der, in der Tabelle auf dem Deckblatt übereinstimmen und bietet somit noch eine weitere Möglichkeit der Kontrolle an.
	
	\section{Vorlage verbessern}
	Nachdem ein erster Prototyp zur Digitalisieren der Klausur-Daten entstanden ist, soll außerdem Prüfungsvorlagen entstehen, die für die Digitalisierung optimiert ist. Dabei sollen Probleme die beim Einscannen der aktuellen Vorlage erkannt wurden, behoben werden. 
	
	\section{Weitere Anmerkungen}
	Ferner soll bei der Lösung von der Anschaffung neuer Technologie und Geräte abgesehen werden. Grund dafür sind neben den Anschaffungskosten, die Idee, dass das Ergebnis des Forschungsprojekts in weiteren Lehr- und Forschungseinrichtung Anwendung finden sollte.

%%%%%%%%%%%%%%% - ANFORDERUNGEN - %%%%%%%%%%%%%%%

\chapter{Anforderungen}
	Es soll eine iOS-App entstehen mit der Klausuren digitalisiert werden können. Genauer müssen die, für die Notenfreigabe relevanten Daten der Klausur, in ein tabellarisches Format gebracht werden. Um unterschiedliche Klausuren oder auch andere Dokumente zu unterstützen, soll das Anlegen von Scan-Vorlagen in der App möglich sein. Hierfür (aber auch zum Einscannen) benötigt die App die Berechtigung für das Kamera-System. Diese muss, bei der ersten Benutzung einmalig erteilt werden.
	
	- Bild Scan-Vorlage: Aus Foto wird Dokument extrahiert und Auf der Seite werden Regionen markiert. -> Sammlung an Seiten mit Regionen
	
	Eine Scan-Vorlagen bestehen aus zwei Komponenten. Die eine sind zugeschnittene Fotos der Seiten eines Dokuments. Und die zweite sind die Regionen auf den Bildern, wo sich die zu digitalisierenden Informationen befinden. Dem Benutzer muss es möglich sein, zuerst die Fotos zu machen und anschließend Regionen auf den Seiten zu markieren. Weiter benötigen die Regionen Namen bzw. Typen, die der Benutzer angeben muss. Die Angabe des Typs ist wichtig, da dadurch eindeutig wird, ob es sich um die Note oder die Matrikelnummer des Studenten handelt. Diese Eindeutigkeit wird nicht nur für die automatische Erstellung der Tabelle benötigt, sondern auch für die Texterkennung. 
	
	Optical character recognition (OCR)\nomenclature{OCR}{optical character recognition, deutsch: optische Zeichen Erkennung und im deutschen Synonym für Texterkennung}, auf deutsch optische Zeichen Erkennung, soll dazu benutzt werden, die Informationen der Regionen zu digitalisieren. Um die Texterkennung zu verbessern, soll der Typ der Regionen verwendet werden. Damit könne dem OCR mögliche Ergebnisse mitgeteilt oder die Ergebnisse angepasst werden. Beispielsweise bestehen Zensuren immer aus Dezimalzahlen von Eins bis Sechs und mit nur einer Nachkommastelle. Wird der Buchstabe O anstelle der Ziffer Null erkannt, kann so der Fehler korrigiert werden.
	
	Zudem muss die Texterkennung auf dem Gerät selbst statt finden. Allerdings darf die App auch Texterkennung auf externen Servern unterstützen. Hierfür müssen dann die entsprechenden API-Schnittstellen implementiert oder für einen eigenen Server entwickelt werden.

	Die digitalisierten Daten, die für die Texterkennung entstandenen Dokumenten-Bilder und die erstellten Scan-Vorlagen, soll außerhalb der App auf einem Server gespeichert werden. Somit ist die Möglichkeit gegeben, die Vorlagen wieder zu verwenden und anderen Benutzern der App zur Verfügung zu stellen. Des Weiteren hat diese zentralen Stelle den Vorteil alle anfallenden Daten zu verwalten, was die Benutzung der App trotz vieler Benutzer vereinfacht.
	
	Damit Synchronität der Daten auf den Geräten und dem Server gewährleistet werden kann, benötig die App hierfür ebenfalls Schnittstellen. Diese sollten den standardmäßigen \textit{CRUD}\nomenclature{CRUD}{Das Akronym CRUD steht für Create/Erstellen, Read/Lesen, Update/Aktualisieren und Delete/Löschen und umfasst die vier grundlegenden Operationen persistenter Speicher}-Operationen entsprechen. 
	
	Da die gesamte Kommunikation über das Internet geschieht, muss das Softwaresystem die üblichen Datenschutz- und Datensicherheit-Richtlinien entsprechen, bzw. implementieren.
	

%%%%%%%%%%%%%%% - GRUND IDEE ZUM LÖSEN DES PROBLEMS - %%%%%%%%%%%%%%%

\chapter{Konzept der Dokumenten-Scanner-App}
	Wie in der Problemstellung und Anforderung beschreiben, soll die Dokumenten-Scanner App wichtigen Daten auf dem Deckblatt einer Klausur, wie Vor- und Nachname des Studenten, seine Matrikelnummer sowie die Note erkennen, digitalisieren und in ein für die Notenfreigabe geeignetes Format bringen. Die digitalisierten Daten sollen anschließend an einen Server gesendet werde, wo sie und die beim Einscannen entstandenen Bilder gespeichert werden. Verwendet eine einzuscannende Klausur die Klausuren-Vorlage der Fakultät CB oder ähnliche, die eine Punkteübersicht haben, dann soll es außerdem möglich sein, die Punkte sowie die Note auf Richtigkeit zu überprüfen. 

	Zur Digitalisierung und Kontrolle der Daten werden Scan-Vorlagen verwendet. Diese muss vor dem Einscannen der ausgefüllten Klausuren erstellt werden, um zu gewährleisten, dass die richtigen Daten digitalisiert werden. Beim Erstellen einer Scan-Vorlage geht man wie folgt vor:
	\begin{itemize}
		\item Zuerst wird jede Seite der Klausur fotografiert. Dabei wird in jedem Bild das Dokument erkannt, vom Hintergrund getrennt oder genauer gesagt ausgeschnitten und perfekt ausgerichtet. Zudem wird der Kontrast erhöht, so dass die Schrift leichter lesbarer wird.
		
			- Bild einfügen von APP
		
		\item Anschließend markiert man diejenigen Regionen auf jeder Seite, die zu digitalisieren und/oder kontrollieren sind. Zusätzlich muss jeder Region ein Name und einer der folgenden Typen zugeordnet werden: Unbekannt, Name, Matrikelnummer, Seminargruppe, Punkte und Note. Mithilfe des Typs wird die Texterkennung verbessert und falsche Ergebnisse automatisch korrigiert. Dazu später mehr.
		
			- Bild einfügen von APP
			
		\item Des Weiteren kann man aus allen markierten Regionen diese auswählen, die in Relation stehen. Diesen Relationen werden danach noch weitere Eigenschaften zugeteilt, um sie später bei der Kontrolle richtig zu analysieren. So eine Eigenschaft könnte sein, dass es sich um einen Vergleich zwischen zwei in Relation stehenden Regionen handelt. Z. B. ist die eine Region die Zelle in der Punkteübersichts-Tabelle, in der die erreichten Punkte zu Aufgabe 1 rein geschrieben werden sollen. Und die zweite Region die Stelle über der Aufgabenstellung von Aufgabe 1, in der ebenfalls die erreichten Punkte eingetragen werden sollen. Weitere Beispiele für solche Relationen und Eigenschaften sind die Summanden für die erreichte Gesamtpunktzahl und die Summe selbst, oder auch noch die daraus resultierende Note.
		
			- Bild einfügen von APP
		
		\item Als letztes speichert man die Vorlage ab, welche dann automatisch an einen Server gesendet wird, so dass andere diese Vorlage ebenfalls nutzen können.
	\end{itemize}
	
	Nach der Erstellung einer Vorlage kann das Dokument digitalisiert werden. Dazu scannt man die Seiten in derselben Reihenfolge, wie in der Scan-Vorlage ein. Die App digitalisiert mithilfe von OCR den Inhalt, in den vordefinierten Regionen. Der Typ der Region nimmt nun Einfluss auf das Ergebnis. Durch ihn wird während der Worterkennungsphase eine Liste an vordefinierten Ergebnissen eingespeist. Diese Liste hat Vorrang vor dem Standard-Lexikon, welches verwendet wird. Ein Beispiel für solch eine Liste könnten, wie in den Anforderungen beschrieben, alle möglichen Noten sein. Auch vorstellbar ist, dass alle Seminargruppen oder Namen von Personen, die an der Klausur teilgenommen haben, dort verwendet werden. 
	
	Weiter wäre eine zusätzliche Möglichkeit der Korrektur erdenklich. Angenommen, jemand verschreibt sich bei seiner Seminargruppe und vergisst ein Zeichen. Die Texterkennung erkennt zwar jedes Zeichen richtig, jedoch wäre es für die Notenfreigabe eine nicht korrekte Seminargruppe. Deshalb könnte man auch im Nachhinein das Ergebnis mit einer Seminargruppe ersetzt, die die größte Übereinstimmung mit dem erkannten Wort hat. Dieses Vorgehen kann man dadurch verbessern, wenn man dafür nur die Seminargruppen-Bezeichnungen verwendet, die auch tatsächlich die Klausur geschrieben haben. Ähnliches gilt auch für die Namen oder Matrikelnummern der Studenten.
	
	Um wirklich sicher zu gehen, ist es aber im Anschluss an die Texterkennung möglich die Resultate noch einmal zu überprüfen und zu korrigieren, bevor es an den Server zum Abspeichern gesendet wird. Dazu wird außerdem die confidence des OCR bei jeder Region angezeigt. Diese sagt aus, wie sicher sich die Texterkennung ist, das das Ergebnis stimmt. Jedoch kann es auch hier falsche Positiv-Ergebnisse geben, weshalb auch das kein perfekter Indikator für die Richtigkeit ist.
	
	Ein wichtiger Vorteil einer App gegenüber anderen Lösungen ist, dass so gut wie jeder Mitarbeiter an einer Lehr- und Forschungseinrichtung ein eigenes oder Zugang zu einem Smartphone oder Tablet hat.

		... noch mehr? (Auf Tobias warten...)
	
	- Noch mehr zu OCR sagen: was es ist, wie es geht, komplett eigenes Kapitel?

	- Upload download verwendung auf allen geräten


%%%%%%%%%%%%%%% - PROTOTYP - %%%%%%%%%%%%%%%

\chapter{Entwicklung des ersten Prototyps}
	Diese Kapitel beschreibt die Entwicklung der iOS-App in einer ähnlichen Reihenfolge, wie das Wasserfall-Model über die Verwaltung der Entwicklung großer Softwaresysteme nach Winston W. Royce. - Quelle:...
	
	\section{Anforderungsplanung}
	Vor Beginn des Praktikums wurde diskutiert und kalkuliert, welches Thema aus dem Projekt Memo Space für ein zwölfwöchiges Praktikum geeignet ist. Dabei entstand eine Art Durchführbarkeits- / Machbarkeitsstudie, welche zur Entscheidung führte, ein Dokumenten-Scanner-Softwaresystem umzusetzen. Aufgrund der Kenntnisse und Erfahrung, soll ein Backend mit entsprechenden Schnittstellen und einer Android-App von Tobias Kallauke umgesetzt werden, währende vom Verfasser eine iOS-App gefordert ist. Die Anwendung soll den Anforderungen, die im gleichnamigen Kapitel zu finden sind, erfüllen. Für weitere Details über den Server und die Android-App siehe im Praktikumsbericht von Tobias Kallauke.

	\section{Planung und Vorbereitung bzw. Analyse und Definition}
		Die Aufgaben bzw. Ziele dieser Phase sind:
		\begin{enumerate}
			\item die Auseinandersetzung mit der Problemstellung bzw. den Anforderungen,
			\item das Betreiben von einer Problemanalyse,
			\item die Entwicklung von Ideen und eines genauen Konzepts der iOS-App, wie im Kapitel Konzept der Dokumenten-Scanner-App zu lesen ist und 
			\item das Entwickeln eines ersten Prototyps, als Machbarkeitsnachweis.
		\end{enumerate}
		
		Für die Entwicklung des Konzepts spielen die Dokumentationen der Frameworks \textit{Vision}, \textit{VisionKit} und \textit{PhotoKit} von \textit{Apple} eine entscheidende Rolle. Aus ihnen wird klar, welche Funktionen der App noch zu entwickeln und welche in Frameworks schon vorhanden sind. Z. B. ist das Erkennen und das gerade Ziehen von Dokumenten in Echtzeit, sowie die Texterkennung auf Bildern in \textit{Vision} und \textit{VisionKit} enthalten. Bei der Entwicklung des ersten Prototyps half eine Beispiel-App von \textit{Apple} \cite{noauthor_detecting_nodate}, die das Erkennen von Objekten in Standbildern mithilfe der genannten Frameworks umsetzt.
		
		Durch die Auseinandersetzung mit den Bibliotheken konnte festgestellt werden, dass die zu entwickelnde App nur Geräte mit iOS 13.0 oder neuer unterstützen werden kann. Grund dafür sind die \textit{Apple} Frameworks \textit{SwiftUI} und \textit{VisionKit}. \textit{SwiftUI} bietet die Möglichkeit, Benutzeroberflächen für alle \textit{Apple}-Plattformen in \textit{Swift} zu erstellen. Die deklarative \textit{Swift}-Syntax ist einfach zu lesen und schnell zu schreiben, so dass es möglich ist, die App in wenigen Wochen für iPhone und iPad zu schreiben. Als Alternative gibt es noch \textit{UIKit} oder auch \textit{AppKit}, die unter alle iOS Versionen funktioniere. Diese Frameworks sind allerdings nicht deklarativ sodass, Views sowohl im Code als auch in Interface-Dateien getrennt voneinander erstellt und konfiguriert werden müssen \cite{sillmann_einstieg_nodate}. Dadurch dauert die Entwicklung einer App, im Gegensatz zu SwiftUI deutlich länger, wie man auch hier in dem Video \footnote{SwiftUI vs UIKit – Comparison of building the same app in each framework - \href{https://www.youtube.com/watch?v=qk2y-TiLDZo}{\url{https://www.youtube.com/watch?v=qk2y-TiLDZo}} } sieht. Sehr ähnliche Erfahrung, wie sie von  hat der Verfasser vor dem Praktikum mit den beiden Frameworks gemacht. \textit{VisionKit} dagegen, ist das Framework zum Scannen der Dokumente. Auch hierfür gibt es eine Alternative. Das Framework ist von den Entwicklern von \textit{WeTransfer} und funktioniert auch auf älteren iOS Versionen. Allerdings unterstützt \textit{WeScan} \footnote{WeScan GitHub Repository - \href{https://github.com/WeTransfer/WeScan}{\url{https://github.com/WeTransfer/WeScan}}} noch kein stapelweises Scannen. Das bedeutet, man kann immer nur ein Foto machen, welches erst abgespeichert werden muss, bevor man das nächste machen kann. Das ist für den Benutzer nicht bequem und spart wahrscheinlich auch keine Zeit.

		Das Scrum-Konzept, welches sich für agile Softwareentwicklung anbietet, wurde zur Projektplanung und zum Projektmanagement verwendet. Als Versionsverwaltung der Software wurde Git in Kombination mit GitHub Issues und GitHub Projects als Planungstools benutzt. So konnte der Verfasser selbständig jedem Sprint Aufgaben zuordnen und den Fortschritt nachvollziehen.

	\section{Grundlagen}
			In den folgenden zwei Absätze werden wichtige Konzepte erklärt, die im Kapitel Entwurf und Design nochmal aufgegriffen werden.

		\paragraph{Model-View-ViewModel:}
			Das Entwurfsmuster Model View ViewModel (MVVM)\nomenclature{MVVM}{Model View ViewModel} entstand bei \textit{Microsoft} im Jahr 2005 mit der \textit{Windows Presentation Foundation} (WPF) und \textit{Silverlight-Technologien}. MVVM verwendet das Konzept eines Schichtmodells und ist eine abstrakte Darstellung einer Benutzeroberfläche, in Form einer Klasse, wie man sie unter objektorientierten Programmiersprachen kennt. Diese Klasse enthält Daten, die auf der Benutzeroberfläche angezeigt werden sollen und Anweisungen, die auf der Benutzeroberfläche aufgerufen werden können. Dieses sogenannte ViewModel, weiß nichts von Views, wie es sonst bei anderen Entwurfsmustern üblich ist, um Daten auf der Benutzeroberfläche anzuzeigen. Stattdessen verwendet eine MVVM-View eine Bindungsfunktion (data binging) zur bidirektionalen Zuordnung von Daten aus dem ViewModel zu den jeweiligen Eigenschaften auf der View. Z. B. Einträge in einer Dropdown-Menü. Aber auch das binden von Daten aus dem Model zu Benutzereingaben durch Maus, Tastatur oder Touch-Screens ist möglich. Beispielsweise kann ein Mausklick eine Anweisung in dem ViewModel auslösen. Diese verändert einen Wert im Model, wodurch die View durch data binding aktualisiert wird. \cite{freeman_pro_2017} \cite{bragge_model-view-controller_2013}

		\paragraph{Redux.js:}
			Redux.js ist eine Bibliothek\footnote{Redux.js Github Repository - \href{https://github.com/reduxjs/redux}{\url{https://github.com/reduxjs/redux}}} für die Programmiersprache JavaScript. Diese stellt einen sogenannten vorhersehbaren Zustandscontainer\footnote{Core Concepts - \href{https://redux.js.org/introduction/core-concepts}{\url{https://redux.js.org/introduction/core-concepts}}} für JavaScript Anwendungen bereit. In diesem Container wird der Zustand der gesamten Anwendung in einem Objektbaum innerhalb eines einzigen Speichers gespeichert. Diesen Baum kann man sich vorstellen, wie eine Matrjoschka die weitere Puppen in sich hat. Der wichtige Unterschied zwischen einer Matrjoschka und einem Baum ist jedoch, dass in eine Puppe genau nur eine andere gesteckt werden kann. Beim einem Objektbaum hingegen können mehrere Objekte in ein Objekt ''gesteckt'' werden und so weiter. Im Bezug auf einen Zustandscontainer enthält dieser dann States (Objekte) auf deutsch Zustände und Daten, die unteranderem auf der Benutzeroberfläche angezeigt werden sollen. Diese Aufteilung in die States ist dazu gedacht, einer View oder mehrere zusammenhängende Views nur die wirklich benötigten Daten bereit zustellen. Diese Struktur erleichtert das Testen oder Untersuchen der Anwendung und ermöglicht es, durch hinzufügen eines neuen States den aktuellen Entwicklungsstand der Anwendung beizubehalten und dadurch den Entwicklungsprozess zu beschleunigen. \\
			Ein weiteres wichtiges Prinzip\footnote{Three Principles - \href{https://redux.js.org/introduction/three-principles}{\url{https://redux.js.org/introduction/three-principles}}} von Redux ist, dass die Daten in den States Schreibgeschütz sind. Die einzige Möglichkeit den Zustand zu ändern, besteht darin, eine Aktion auszulösen, die beschreibt, was passiert. Dadurch wird sichergestellt, dass weder die Views noch Netzwerk-Rückrufe jemals direkt an den Zustand schreiben werden. Stattdessen bringen sie nur die Absicht zum Ausdruck, den Zustand zu verändern und lösen eine Aktion aus. Da alle Änderungen zentralisiert sind und eine nach der anderen in einer strikten Reihenfolge erfolgen, gibt es weniger Programmfehler-Quellen.

	\section{Entwurf und Design}
		
		Nach der Planung und Vorbereitung begann ich damit, mir Gedanken zu dem Workflow der App zu machen und erstellte den ersten Sprint bis zum nächsten Meeting. Für den Workflow fertigte ich zu jeder View ein grobes Design an, welches sehr schnell in \textit{SwiftUI} umzusetzen ist. Das Aussehen der App sollten sich dann im Laufe der Zeit noch ändern, jedoch stand erstmal die Umsetzung des Konzepts im Vordergrund. 

		Aus den Designs und dem Workflow heraus entwickelte ich einen groben Plan, wie die Daten in der App gehandhabt werden sollten. Da \textit{SwiftUI} das Entwurfsmuster MVVM umsetzt, überlegte ich mir eine Struktur für das ViewModel und für den Datenfluss der asynchronen Server-Rückrufe. Jedoch stellte ich schnell fest, dass meine Idee nicht ideal ist, wenn noch mehr Views hinzukommen. Aus diesen Gründen suchte ich nach einer besseren Lösung. Dabei entdeckte ich die JavaScript-Bibliothek \textit{Redux.js} zur Verwaltung von Zustandsinformationen in Webanwendung. 
		
		\textit{Redux} hilft durch das Konzept, komplexen Views mit vielen Daten schnell und einfach zu implementieren. Auch werden so Serverantworten und zwischengespeicherte Daten sowie lokal erstellte Daten, die noch nicht auf dem Server gespeichert wurden, strukturiert und zentral abgelegt. Das erleichtert nicht nur das Wiederverwenden von Daten, sondern spart auch wiederholte Server-Aufrufe ein. Für mich war es wichtig, trotz der der begrenzten Zeit im Praktikum, möglichst viel mit wenig Fehlern umzusetzen. Und auch da sah ich in Redux großes Potential mir dabei zu helfen. Denn die modulare Aufteilung des State Containers ermöglicht eine schnelle Weiterentwicklung der App, auch wenn die Komplexität der Anwendung steigt. Und die Vorteile von Redux hinsichtlich des Testens und Untersuchens der App sollten helfen, Fehler möglichst schnell ausfindig zu machen, trotz asynchroner Programmabschnitte.
		
		Daher lag es nah die wesentlichen Konzepte von \textit{Redux} umzusetzen und einen \textit{Redux} ähnlichen State Container, als ViewModel zu implementieren. Ich überlegte mir noch vor Beginn der Implementierung die States sowie deren Aktionen, die zwingend notwendig sind, um Views und Daten sinnvoll zu separieren.
		
		
		
		Der Container sollte mindestens 5 States haben: 
		\begin{itemize}
			\item für Authentifizierung sowie Registrierung,
			\item für das Anlegen von Vorlagen, um Zwischenergebnisse zu speichern,
			\item für das Ausführen von Server-Aufrufen zum Speichern und Abrufen von Templates, 
			\item für die Steuerung von in Abhängigkeit stehenden Views und
			\item für sonstige Daten, die sehr häufig verwendet werden.
		\end{itemize}
		
		
			
		Da ich zuvor noch nie mit einem State Container gearbeitet habe, fing ich 
		
		
		
		Ich machte mir Gedanken darüber, welche States zu implementieren wären und kam auf 
		Die App benötigte für 
		
		- auth
		
		- service
		
		- template anlegen
		
		- routing
		
		- sonstiges
		
		- actions können actions aufrufen -> verändern den container und die views upadten sich, zu den änderungen in den states

	\section{Implementierung}
		
		- In welcher Reihenfolge soll ich hier vor gehen? In der logischen oder in der, wie ich es gemacht habe?
		
		Bei der Implementierung und Umsetzung der Anforderungen 
		
		
		
		
		Einzelne Schritte als eigene Section mit Verlauf der jeweiligen Entwicklung: erst das, dann kam das, dann musste es so, ...
		
		\subsection{Registrieren und Anmelden}
			
		\subsection{Template erstellen}
			- Workflow als Bild?
		
		\subsection{Template speichern}
		
		\subsection{Template verwenden}
		
		.
		
		Die Beispiel-App von \textit{Apple} nahm ich mir als Vorbild für die Umsetzung der Texterkennung.
		
		...
		
		- keine Unit Tests etc. keine Zeit
		
	\section{''Abnahme''}	
		Abnahme ... \\ \\ \\
	
	
	Weitere Punkte die in Kapitel 6 rein könnten/sollten: 
	
	- beschreiben wie die einzelnen Views/Seiten nun aussehen und funktionieren? oder eher den Prozess der Entwicklung?
	
	- Bild mit Wireframe aller Views und deren Workflow?
	
	- AppState erklären? (Single source of truth) -> später AppStore mit mehreren States
	
	- erklären deskew / Dokument ausrichten?
	
	- Umrechnung der Bilder vom Template aus und zuschneiden der neuen Regionen erklären
	
	- Attribute hinzufügen (den Vorgang) -> Rechteck einzeichnen...
	
	- Verwendete Programme, Sprache, etc. 
	
	- ein paar Worte, wie gut die Entwicklung lief (Simulator vs. echtes Gerät), wo sind die Grenzen des Simulators (Keine Kamera), wo waren Probleme mit dem echten Gerät...

%%%%%%%%%%%%%%% - WEITERENTWICKLUNG - %%%%%%%%%%%%%%%

\chapter{Weitere Entwicklung und Besonderheiten}

	\section{App-Architektur}
		- redux like app store mit states erklären? (single source of truth)
	
	\section{API}
		- auf die API eingehen? oder nur Tobias?
	
	\section{}
		- 
%%%%%%%%%%%%%%% - GRENZEN - %%%%%%%%%%%%%%%

\chapter{Grenzen der App}
	\section{Probleme beim Erkennen von Dokumenten}
		- Gleich-farbiger Hintergrund
			
		- Wenig Licht
		
		- Hintergrund mit starken Kanten
		
		- Runde Ecken, keine Ecken
		
		- Starke Kanten im Bild (schwarzer Kreditkartenstreifen)
		
		- zu jedem möglichem vlt. dann ein Beispiel Bild
		
	\section{Probleme der Klausur-Vorlage beim Scannen}
		- Schrift zu klein
		
		- zu wenig Platz
		
		- zu viel Platz
		
		- Überschneidungen
		
		- ...
		
	\section{Weitere ...}
		- Abstürze, Memory Leaks, 



%%%%%%%%%%%%%%% - TEMPLATES - %%%%%%%%%%%%%%%

\chapter{Templates}
	- Template und den Entwicklungsprozess vorstellen, 
	
	- wieso weshalb warum muss das nun so aussehen?
	
	- Was kann an der neuen Vorlage immer noch optimiert werden?	
	
	- Welche Probleme konnten behoben werden?
	
%%%%%%%%%%%%%%% - AUSBLICK - %%%%%%%%%%%%%%%
		
\chapter{Ausblick}
	- Es müssen nicht nur Klausuren sein, sondern alles mögliche (Krankenscheine, Urlaubsscheine, was auch immer, nur DB muss angepasst werden. (oder automatisches generieren von DB-Tabellen an Hand der Template Attribute))
	
	- Server mit Bildern als Klausuren-Einsicht nutzen -> allerdings viele Probleme (Klausuren würden kopiert werden -> Profs mehr Arbeit, keinen direkten Kontakt zum Prof wegen Fragen, Verbesserungen oder Anmerkungen, ... )
	  (warum genau, sollen die gespeichert werden? meine/unsere Idee Online Klausureneinsicht -> ins Fazit/Ausblick (hat viele ''Probleme'' ))
	
	- QR-Code-Idee um Vorlagen weg zu lassen -> Programm oder Plugin (Word/LaTeX) was die QR-Codes dann automatisch erstellt und richtig einfügt. (Wichtige Daten sind da hinterlegt, Vor- und Nachteile von QR-Codes, ...)
	
	- Ist die App nun schneller als der normale Umgang bleibt offen -> Bachelorarbeit knüpft da an...
	

\Anhang

\chapter{Tätigkeitsbericht}
	\textbf{24.02. - 01.03.}
	Ich habe mich mit der Problemstellung auseinander gesetzt, Ideen gesammelt, Problemanalyse betrieben und einen kleinen Prototypen entwickelt. Dazu erstellte ich einen minimalen Projektplanung, arbeitete mich in die Frameworks \textit{Vision} und \textit{VisionKit} ein und setzte eine Versionsverwaltung auf. Zusätzlich suchte ich nach einer passenden App-Architektur, die geeignet für das deklarative GUI-Framework SwiftUI, sowie für asynchrone Aufgaben, wie z. B. API-Aufrufe ist. Dabei stieß ich auf \textit{Cleancode Architecture} und \textit{Redux}.
	
	\textbf{02.03. - 08.03.} 
	In dieser Woche habe ich die Texterkennung auf den berechneten Regionen eines neuen Fotos implementiert, den Workflow sowie viele andere Kleinigkeiten in der App verbessert und alle Fehler der letzten Woche behoben, sodass ich neue Dinge implementieren konnte. Zudem probierte ich CI sowie Lint für das Projekt aus. Da CI für eine iOS-App mit \textit{GitHub Actions} schwer aufzusetzen war und ab April etwas kosten würde, lies ich es sein. Des Weiteren pflegte ich das Projekt Management durch \textit{Issues} und \textit{Project-Boards} in GitHub. Anschließend programmierte ich den App-Workflow so um, dass nun mehr als eine Seite aufgenommen und analysiert werden konnte. \\ 			
	Abgesehen von neuem Quellcode fing ich an den Praktikumsbericht zu schreiben und arbeitet mich dafür in \LaTeX \xspace und die Bachelorarbeit-Vorlage für \LaTeX \xspace der Hochschule Mittweida ein.
	
	\textbf{09.03. - 15.03.} 
	Zu Beginn der dritten Woche schaute ich mir Möglichkeiten für serverseitiges OCR an. Genauer sammelte ich Informationen zu dem Framework Vapor und Swift unter Linux. Jedoch funktionieren die Frameworks \textit{Vision} und \textit{CoreML} von \textit{Apple} unter Linux nicht, weshalb sich IronOCR als beste Option herausstellte. Ich entwickelte ein Datenbankmodell, mithilfe der in der App verwendeten Datentypen und erstellte dazu noch eine JSON-Struktur die später für die APIs verwendet werden könnte. Außerdem gab es ein Meeting, in dem wir unseren aktuellen Stand präsentieren sollten, um weitere Schritte und Aufgaben zu planen.Bis zum Ende der Woche arbeitete ich weiter an meinem Beleg und schrieb den Datenfluss in der App um. Nun ähnelt er sehr dem Redux-Model.
	
	\textbf{16.03. - 22.03.} 
	Anfangs habe ich weiter an meinem Praktikumsbericht geschrieben, neue Issues hinzugefügt und bearbeitet. Außerdem gepflegte ich die Dokumentation und betrieben Projekt Management, um nun Links zwischen Regionen hinzuzufügen. Dabei entstanden neue Views und der Redux-Store musste dadurch angepasst werden. Es kam eine Erweiterung für die Texterkennung hinzu, so dass man durch die Auswahl eines Datentyps, das Resultat der Erkennung verbessern konnte. Des Weiteren habe ich bis zum Ende der Woche die Vergleich-Links vollständig implementiert und die App auf Fehler und Abstürze kontrolliert, sowie den Beleg um einige Kapitel erweitert.
	
	\textbf{23.03. - 29.03.} 
	Ich begann den Workflow und die Navigation in der App, zu verbessern und vereinfachen. Dabei beseitigte ich Quellcode des Prototyps, erweiterte die Dokumentation und behob einige Fehler. Anschließend überarbeitet ich einige Views, so dass sie übersichtlicher und einfacher zu benutzen sind. Nach dem iOS 13.4 Update in der Mitte der Woche, funktionierte ein Teil der App nicht, da sich das Verhalten von Views geändert hat. Ich behob die Fehler, testete ausgiebig die App und fügte iPad Unterstützung hinzu. Des Weiteren entstand ein neues verbessertes Template und ich schrieb einen großen Teil am Bericht.
	
	\textbf{30.03. - 05.04.} 
	Das Backend für die App war soweit, dass ich es aufsetzten und die API-Schnittstellen implementieren konnte. Dazu erstellte ich Views für Registrieren und Anmelden, die mithilfe von regulären Ausdrücken, die Eingaben überprüfen. Außerdem entwickelte ich einen Schicht im App-Store, für die anfallenden asynchronen Aufgaben. Dabei laß ich mich in das Framework Combine ein und überlegte mir einen geeigneten Aufbau. Da der Ansatz von Combine sehr neu für mich war, dauerte es zwei Tage, bis ein erster API-Service mit Fehler-Handling funktionierte. Zum Ende der Woche waren alle der Create-Schnittstellen implementiert, getestet und dokumentiert. Nebenbei erstellte ein paar Issues für das Backend und sprach mich mit Tobias über OCR auf dem Server ab.
	
	\textbf{06.04. - 12.04}

\bibliography{Beleg}
\bibliographystyle{iso690.bst}


%\begin{thebibliography}{99}
%	
%	\bibitem{hsmwPortrait}
%	Hochschule Mittweida, Portrait[online]. URL https://www.hs-mittweida.de/hochschule/portrait.html? . - abgerufen am 2020-04-15
%
%	\bibitem{digital}
%		% nochmal genauer nachschauen, guten Link finden, und genauen Namen...
%		Graf, Michael Partner bei PwC
%		\\\url{http://www-cs-faculty.stanford.edu/~{}uno/abcde.html}
%		
%	\bibitem{visionBasicApp}
%		Detecting Objects in Still Images | Apple Developer Documentation. URL \href{https://developer.apple.com/documentation/vision/detecting_objects_in_still_images}{\url{https://developer.apple.com/documentation/vision/detecting_objects_in_still_images}} . - abgerufen am 12.04.2020
%
%	
%	\bibitem{bragge2013model}
%		Bragge, Matti: Model-View-Controller architectural pattern and its evolution in graphical user interface frameworks (2013), S. 34
%		
%	\bibitem{freeman}
%		Freeman, Adam: Pro ASP.NET Core MVC 2, Expert’s Voice in .NET. Apress, 2017 — ISBN 978-1-4842-3150-0
%
%		
%	\bibitem{heise2019einsitegInSwiftUI}
%		Sillmann, Thomas: Einstieg in SwiftUI. URL \href{https://www.heise.de/developer/artikel/Einstieg-in-SwiftUI-4594018.html}{\url{https://www.heise.de/developer/artikel/Einstieg-in-SwiftUI-4594018.html}}. - abgerufen am 12.04.2020. — Developer
%		
%	
%\end{thebibliography}

\end{document}

%	Gemeinsam mit den Betreuern des Praktikums und Tobias Kallauke entstand eine typische Client-Server-Architektur. Mit Hilfe von verschiedenen Apps, auf der Basis von Android, iOS oder auch im Browser sollen.
	
%	Die Grund Idee des Dokumentenscanner ist es den Vorgang der Klausuren-Kontrolle und deren Noteneintragung zu vereinfachen. Genauer soll es möglich sein die wichtigen Daten auf der ersten Seite einer Klausur, wie Vor- und Nachname des Studenten, seine Matrikelnummer sowie die Note zu erkennen, digitalisieren und in ein geeignetes Format zu bringen, um es anschließend der Notenfreigabe der Hochschule zu übermitteln. 

%	Des Weiteren ist es für die Prüfer der Klausur hilfreich, wenn die Punkte der Klausur kontrolliert bzw. überprüft werden. Konkret bezieht sich die Überprüfung der Punkte auf eine Klausuren-Format-Vorlage, die im Bereich der Fakultät Angewandte Computer- und Biowissenschaften genutzt wird. Diese Vorlage hat auf dem Deckblatt der Klausur zu jeder Aufgabe ein Feld für die erreichten und die zu erreichenden Punkte. Sowie die Summe der erreichten Punkte. Hier könnte der Dokumentenscanner die erreichte Gesamtpunktezahl sowie die daraus resultierende Note überprüfen. Des Weiteren befinden sich auf den nachfolgenden Seite der Klausur über jeder Aufgabe ein Feld für die erreichten und die zu erreichenden Punkte der jeweiligen Aufgabe. Die dort vom Prüfer eingetragenen Punkte könnten ebenfalls mit denen auf dem Deckblatt verglichen werden. 
%
%	Durch die Erweiterung der Grundidee um Scan-Vorlagen wurde es auch möglich Daten anderer Dokumente zu digitalisieren und in ein geeignetes Format zu bringen. Genauer soll es dem Benutzer möglich sein von jedem beliebigen Dokument eine Vorlage zu erstellen, in der er Stellen auf dem Dokument markiert, wo sich die zu digitalisierenden Daten befinden.

%	\begin{longtable}[h!]{ p{4em} | p{356pt} }
%		Datum & Tätigkeit \\ \hline
%		\endfirsthead
%		Datum & Tätigkeit \\ \hline
% 		\endhead
%
% 		\hline
% 		\endfoot
%		
%		24.02. - 01.03. & 
%			Ich habe mich mit der Problemstellung auseinander gesetzt, Ideen gesammelt, Problemanalyse betrieben und einen kleinen Prototypen entwickelt. Dazu erstellte ich einen minimalen Projektplanung, arbeitete mich in die Frameworks \textit{Vision} und \textit{VisionKit} ein und setzte eine Versionsverwaltung auf. Zusätzlich suchte ich nach einer passenden App-Architektur, die geeignet für das deklarative GUI-Framework SwiftUI, sowie für asynchrone Aufgaben, wie z. B. API-Aufrufe ist. Dabei stieß ich auf \textit{Cleancode Architecture} und \textit{Redux}.
%				
%			Am Ende der Woche konnte man in der App Vorlagen mit einer Seite erstellen. Das heißt man konnte ein Foto machen, aus welchem das Dokument rausgeschnitten und anschließend ausgerichtet wurde. Weiter war es möglich Regionen auf dem Dokument zu markieren und diese in der Vorlage abspeichern. Ansonsten konnten die Vorlage schon dazu benutzt werden, um die Regionen auf dem neuen Foto heraus zu schneiden.
%			\\ \hline
%		02.03. - 08.03. &
%			In dieser Woche habe ich die Texterkennung auf den berechneten Regionen eines neuen Fotos implementiert, den Workflow sowie viele andere Kleinigkeiten in der App verbessert und alle Fehler der letzten Woche behoben, sodass ich neue Dinge implementieren konnte. Zudem probierte ich CI sowie Lint für das Projekt aus. Da CI für eine iOS-App mit \textit{GitHub Actions} schwer aufzusetzen war und ab April etwas kosten würde, lies ich es sein. Des Weiteren pflegte ich das Projekt Management durch \textit{Issues} und \textit{Project-Boards} in GitHub. Anschließend programmierte ich den App-Workflow so um, dass nun mehr als eine Seite aufgenommen und analysiert werden konnte. 
%				
%			Abgesehen von neuem Quellcode fing ich an den Praktikumsbericht zu schreiben und arbeitet mich dafür in \LaTeX \xspace und die Bachelorarbeits-Vorlage für \LaTeX \xspace der Hochschule Mittweida ein.
%			\\ \hline
%	 	09.03. - 15.03. &
%	 		Zu Beginn der dritten Woche schaute ich mir Möglichkeiten für serverseitiges OCR an. Genauer sammelte ich Informationen zu dem Framework Vapor und Swift unter Linux. Jedoch funktionieren die Frameworks \textit{Vision} und \textit{CoreML} von \textit{Apple} unter Linux nicht, weshalb sich IronOCR als beste Option herausstellte. Ich entwickelte ein Datenbankmodell, mithilfe der in der App verwendeten Datentypen und erstellte dazu noch eine JSON-Struktur die später für die APIs verwendet werden könnte. Außerdem gab es ein Meeting, in dem wir unseren aktuellen Stand präsentieren sollten, um weitere Schritte und Aufgaben zu planen.Bis zum Ende der Woche arbeitete ich weiter an meinem Beleg und schrieb den Datenfluss in der App um. Nun ähnelt er sehr dem Redux-Model.
%	 		\\ \hline
%	 	16.03. - 22.03. & 
%	 		Anfangs habe ich weiter an meinem Praktikumsbericht geschrieben, neue Issues hinzugefügt und bearbeitet. Außerdem gepflegte ich die Dokumentation und betrieben Projekt Management, um nun Links zwischen Regionen hinzuzufügen. Dabei entstanden neue Views und der Redux-Store musste dadurch angapsst werden. Es kam eine Erweiterung für die Texterkennung hinzu, so dass man durch die Auswahl eines Datentyps, das Resultat der Erkennung verbessern konnte. Des Weiteren habe ich bis zum Ende der Woche die Vergleich-Links vollständig implementiert und die App auf Fehler und Abstürze kontrolliert, sowie den Beleg um einige Kapitel erweitert.
%	 		\\ \hline
%	 	23.03. - 30.03. &
%	 		...
%	 		\\ \hline
%	\end{longtable}

	- wie benutze ich die App wie installiere ich die
	- programmier sprachen, IDEs, 
	- 


